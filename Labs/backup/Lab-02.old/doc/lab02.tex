% !TEX root = ./ICP3038_Lab_02.tex

\section*{Laboratory 2: Introduction to Classes in C++}

%This is a typical script that you will be working with at each
%laboratory session. To work with these scripts efficiently, follow the
%guidelines below.
%\begin{enumerate}
%  \item The script is not a step-by-step tutorial. It introduces the
%    problem but it is your task to solve it.
%
%  \item If you get stuck, make sure that you read \emph{Help} section
%    at the end of the document.
%
%  \item If you still have problems, ask the lecturer, demonstrator or
%    fellow students for help.
%
%  \item Try to do as much work as possible in the class, where it is
%    easier to get help.
%    
%  \item If you need help when working at home, use the Blackboard
%    discussion board.
%    
%  \item Finish all assignments at least a week before the deadline. If you get
%    stuck, you will still have one week to ask for help.
%\end{enumerate}

The aims of today's lab are:
\begin{itemize}
\item Finish Lab~1; %(the Bubble Sort);
\item Understand the constructor in C++ and using initialisation lists;
\item Get familiar with the implementation of (basic) classes;
\item Overwrite some operators (\verb+=+, \verb+<<+ and \verb+>>+);
\item Use \verb+std::string+ instead of \verb+char*+.
\end{itemize}

Next week, we will use the STL library and write our own template class.

\section*{Task 0: Last week}

Make sure you finish your Bubble Sort implementation!
You can find a good example of Bubble Sort execution on Wikipedia:
\url{https://en.wikipedia.org/wiki/Bubble_sort}
There is an animated GIF file.

\section*{Task 1: Using CMake}

Same as last week, we will use CMake to make our lifes easier. So create your MS Visual Studio solutions, Xcode projects, or Unix Makefiles using CMake.
%In the lecture, you saw yesterday that it can be extremely complicated to use the IDE to maintain the project files, etc. particularly if students use Windows, Mac, or Linux. 
%There is a ZIP file on Blackboard with 4 (almost) empty C++ source files. There is also a `CMakeLists.txt' file. 
%It is a good practice to organise your source code using directories, etc. Do not compile code in the same directory as your source code. It can get messy... 
%Extract the ZIP file, and set up the compilation environment using \verb+cmake+. 
%In the GUI, the source directory corresponds to the direction in which `CMakeLists.txt' is. 
%The binary directory is where the code will be compiled. Often, we call this directory `bin'. 
%Once the paths are set, press `Configure'. 
%The first time you run the configuration tool, you have to select a generator.
%If you want to use MSVC++ in the lab, make sure to use \emph{Visual Studio 14 2015 Win64}. 
%For Mac OS X, you may want to use the Xcode generator. 
%For Linux, you may choose Makefile. 
%Once the configuration step is over, press `Generate'. 
%The project files are now ready in the `bin' directory. 
%You can compile the code using the preferred IDE.

\section{Task 2: Illustrating the usefulness of initialisation lists}

For this task, you are given three C++ files:
\begin{enumerate}
  \item \verb+include/Integer.h+, the header file defining a class to handle integer numbers;
  \item \verb+src/Integer.cpp+, the source file that implements the code;
  \item \verb+src/TestInteger.cpp+, a test program to try the class.
\end{enumerate}
We want to adopt good C++ practice. 
There are plenty of tutorials, forums, etc. on the Web that deal with how to write simple C++ classes. 
I googled `How to write a simple class in C++?'. 
The first item returned from Google was: \url{http://stackoverflow.com/questions/865922/how-to-write-a-simple-class-in-c}
The answers to the question on StackOverflow are mostly correct, but not perfect. 
Most people did not use initialisation lists. 
%Today, we will demonstrate how bad some C++ code from the Web can be. 

Initialisation lists have been around for a very long time, but many people still ignore them despite their advantages. 
Initialisation lists are used in constructors to set the initial values of class members in an efficient manner. 
On the web, we may often see:
\begin{lstlisting}
DummyClass my_instance;
my_instance = 10;
\end{lstlisting}
%or
%\begin{lstlisting}
%DummyClass my_instance = 10;
%\end{lstlisting}
%
%Both are equivalent and extremely bad. 
This is not very good. 
What will happen in practice?
\begin{enumerate}
\item One call of the default constructor (\verb+DummyClass::DummyClass()+);
\item One call of the copy operator (\verb+DummyClass::operator=(int)+)
\end{enumerate}
i.e.~two function calls will be performed to declare and initialise the object. 
It is obviously not very good if we consider how many times we may have to create objects in a large program. 
Instead, we should use:
\begin{lstlisting}
DummyClass my_instance(10);
\end{lstlisting}
or
\begin{lstlisting}
DummyClass my_instance = 10;
\end{lstlisting}

Both are equivalent and only the constructor (\verb+DummyClass::DummyClass(int)+) will be called. 
This is more effective and it is better OOP practice. 
%Note that some compilers will automatically interpret
%\begin{lstlisting}
%DummyClass my_instance = 10;
%\end{lstlisting}
%as
%\begin{lstlisting}
%DummyClass my_instance(10);
%\end{lstlisting}

Have a look at the code an execute it. 
3 instances of Integer have been created and data has been printed in the standard output. 
Look at the different values of \verb+m_data+ that have been printed and try to understand what happened.

%Create a CMakeLists.txt file to generate a project for MS VC++ or Xcode (or whatever you may want to use). 
For Task~1, you only have to modify \verb+Integer.cpp+. 
The methods that you need to modify are:
\begin{verbatim}
Integer::Integer()
Integer::Integer(int anInteger)
\end{verbatim}
%\large\textbf{You need to identify what is happening.} To do this, have a look at the output in the console. 

But first, look below at the code of the copy constructor. 
It has an initialisation list. 
See the `\verb+:+' character at the end of Line~2 and look at Line~4 to see how to set the initial value of \verb+m_data+.
\begin{lstlisting}[numbers=left]
//-----------------------------------------
Integer::Integer(const Integer& anInteger):
//-----------------------------------------
        m_data(anInteger.m_data)
//-----------------------------------------
{
    std::cout << "IN: Integer::Integer(const Integer& anInteger)" << std::endl;

    std::cout << "\t" << m_data << std::endl;

    std::cout << "OUT: Integer::Integer(const Integer& anInteger)" << std::endl;
}
\end{lstlisting}

\large\textbf{Add an initialisation list to the other constructors}
(see \verb+src/Integer.cpp+ for further instructions).

The code of the initialisation list is execute first, at the instanciation of the objet. 
Then the body of the constructor is executed. 
It is not unusual for C++ constructors to have an empty body. 
Using the list is faster than using the copy \verb+operator=+ in the body of the constructor. 


\section*{Task 3: Create your own class}

You are going to write a new class called ``StringInverter''. 
You need 3 extra files:
\begin{enumerate}
  \item \verb+include/StringInverter.h+, the header file defining a class to handle integer numbers;
  \item \verb+src/StringInverter.cpp+, the source file that implements the code;
  \item \verb+src/TestStringInverter.cpp+, a test program to try the class.
\end{enumerate}
First, modify CMakeLists.txt to add a new project. 
Just add:
\begin{lstlisting}
add_executable(Task2 
	include/StringInverter.h 
	src/TestStringInverter.cpp 
	src/StringInverter.cpp)
\end{lstlisting}

Now you can start implementing the class. It should include the public methods as follows:
\begin{lstlisting}
    StringInverter();
    StringInverter(const StringInverter& aString);
    StringInverter(const char* aString);
    StringInverter(const std::string& aString);
    ~StringInverter();
    void setString(const char* aString);
    void setString(const std::string& aString);
    const std::string& getString() const;
    const std::string& getInvertedString() const;
    StringInverter& operator=(const StringInverter& aString);
    StringInverter& operator=(const char* aString);
    StringInverter& operator=(const std::string& aString);
    bool operator==(const StringInverter& aString) const;
    bool operator==(const char* aString) const;
    bool operator==(const std::string& aString) const;
    bool operator!=(const StringInverter& aString) const;
    bool operator!=(const char* aString) const;
    bool operator!=(const std::string& aString) const;
\end{lstlisting}

It should also include the following private method:
\begin{lstlisting}
void invertString();
\end{lstlisting}


Also add friend functions \verb+operator<<+ and \verb+operator>>+:
\begin{lstlisting}
std::ostream& operator<<(std::ostream& aStream, const StringInverter& aString);
std::istream& operator>>(std::istream& aStream, StringInverter& aString);
\end{lstlisting}

In \verb+TestStringInverter.cpp+, you have to test each method and each function. 
