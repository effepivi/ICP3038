\documentclass[english,a4paper,12pt,oneside]{article}


%\includeonly{lab4}

%Drafting options
%uncomment for double spacing
%\doublespacing

% \usepackage{acronym}
\usepackage{times}
\usepackage{setspace} 
\usepackage{amsmath}    % need for subequations
\usepackage{graphicx}   % need for figures
%\usepackage{picture}
% \usepackage{wrapfig}
\usepackage{graphics}
 \graphicspath{{./}{../data/}}
 \usepackage{epstopdf}
\usepackage{color}
\usepackage{listings}
\lstset{basicstyle=\ttfamily\footnotesize,
	breaklines=true,
%    basicstyle=\ttfamily,
    keywordstyle=\color{blue}\ttfamily,
    stringstyle=\color{red}\ttfamily,
    commentstyle=\color{magenta}\ttfamily,
%    morecomment=[l][\color{magenta}]{\#}
}
\usepackage{verbatim}   % useful for program listings
\usepackage{color}      % use if color is used in text
\usepackage{subfigure}  % use for side-by-side figures
\usepackage{varioref}
\usepackage{anysize}
\usepackage{natbib}
\usepackage{fancyhdr}
% \usepackage{units}
\usepackage{longtable}
%\usepackage{bbding}
%\usepackage{aeguill}
\usepackage[hyphens]{url}
\usepackage{hyperref}

\setlength{\parskip}{8pt plus 2pt minus 2pt}
\setlength{\parindent}{0pt}

\marginsize{2cm}{2cm}{2cm}{2cm}
\fancypagestyle{plain}{%
  \fancyhf{}%
  \renewcommand{\headrulewidth}{0pt}%
  \renewcommand{\footrulewidth}{0pt}%
}

\pagestyle{fancy}
%\renewcommand{\sectionmark}[1]{\markright{\thesection.\ #1}}

\renewcommand{\sectionmark}[1]{\markright{#1}{}}
\renewcommand{\subsectionmark}[1]{\markright{#1}{}}
\renewcommand{\subsubsectionmark}[1]{\markright{#1}{}}

%\renewcommand{\quote}[1]{\textit{\begin{quote}#1\end{quote}}}
\newcommand{\bold}[1]{\emph{\textbf{#1}}}

\newcommand{\varentry}[1]{{\guillemotleft}\emph{#1}{\guillemotright}}
\newcommand{\code}[1]{{\tt #1}}

\headheight 10mm


\rhead{29/01/2019}
\chead{}
\lhead{ICP3038 --- Computer Vision}
\rfoot{}
\cfoot{- \thepage  \,\,-}
\lfoot{}

\renewcommand{\headrulewidth}{0.4pt}
\renewcommand{\footrulewidth}{0.4pt} 



%%%%%%%%%%%%%%%%%%%%%%%%%%%%%%%%%%%%%%%%%%%%%%%%%%%%%%%%%%%%%%%%%%%%%%%%%%%%%%%%
\begin{document}


%%%%%%%%%%%%%%%%%%%%%%%%%%%%%%%%%%%%%%%%%%%%%%%%%%%%%%%%%%%%%%%%%%%%%%%%%%%%%%%%
\section*{Laboratory 7: Introduction to OpenCV}


The aims of today's lab are:
\begin{itemize}
	\item Install OpenCV;
	\item Create a project using CMake;
	\item Display an image using OpenCV;
	\item Convert from RGB to luminance;
	\item Convert from \verb|unsigned char| to \verb|float|;
	\item Save an image into a file;
	\item Apply some basic filtering techniques.
\end{itemize}

To achieve these goals, we will create several programs:
\begin{enumerate}
	\item \verb+displayImage.cxx+:   A simple program using OpenCV to open an image and display it in a window;
	\item \verb+rgb2grey.cxx+:       A program to convert a RGB image in a greyscale image using OpenCV;
	\item \verb+meanFilter.cpp+:     A program to perform the mean filter using OpenCV;
	\item \verb+logScale.cxx+:       A program to display an image in the log scale;
	\item \verb+medianFilter.cpp+:   A program to perform the median filter using OpenCV;
	\item \verb+gaussianFilter.cpp+: A program to perform the Gaussian filter using OpenCV.
\end{enumerate}

Some test images are provided in the \verb+images+ directory.


%%%%%%%%%%%%%%%%%%%%%%%%%%%%%%%%%%%%%%%%%%%%%%%%%%%%%%%%%%%%%%%%%%%%%%%%%%%%%%%%
\newpage
\section{Installing OpenCV}
\label{sec:Installing OpenCV}
Before using OpenCV, you have to make sure it is installed on the machine you are using. 
\textbf{If it is a University's PC, we will consider that it is installed (so skip this section).}

%done and you can ignore this section. 
%If not OpenCV has to be installed. 
\begin{itemize}
	\item On Mac, OpenCV is available via Homebrew (\verb+brew install opencv3+), Fink and Macports;
 \item Most Linux distributions have packages for OpenCV. Make sure you install the \verb+-devel+ package;
 \item There are pre-compiled versions on \url{http://www.opencv.org/} that can be used on Windows. 
	 \begin{enumerate}
	\item \textbf{Go to OpenCV's website} at \url{http://www.opencv.org/}. 
	\item \textbf{Download the latest version of OpenCV 3} (i.e.~3.4.5): On the top (see Figure~\ref{fig:main_page}), there is a link called `RELEASES', or just click on \url{https://sourceforge.net/projects/opencvlibrary/files/3.4.5/opencv-3.4.5-vc14_vc15.exe/download}. 
		\begin{figure}[tbp]
			\centering
			\includegraphics[width=\textwidth]{opencv_mainpage.png}
			\caption{\label{fig:main_page}OpenCV's website.}
		\end{figure}
%	\item \textbf{Move the file} on \verb+C:+ drive using the File Explorer to avoid using the network (to speed-up the process). 
	\item \textbf{Run the file} and extract the library in \verb+C:+ drive. % (see Figure~\ref{fig:c drive}). 
	%\begin{figure}[tbp]
		%\includegraphics[width=\textwidth]{C_drive.png}
		%\caption{\label{fig:c drive}Installing OpenCV on the C: drive.}
	%\end{figure}

	\item \textbf{Job done:} There is a new directory called \verb+opencv+ in the \verb+C:+ drive.  
		 \end{enumerate}
\end{itemize}



\newpage


\section{Using CMake}

As some of you will use
\begin{itemize}
 \item MS Windows on the University's computers,
 \item MS Windows on their own computer(s),
 \item Mac OS X, and
 \item Linux
\end{itemize}
it is important to keep in mind portability and we saw that toolchains can help us achieve it. 
CMake is an example of toolchain that is user friendly. 

I provided a \verb+CMakeLists.txt+ file that will work on most platforms. 
\lstinputlisting[language=bash,caption=CMakeLists.txt.,label=lst:cmake]{../src/CMakeLists.txt}

Configuring the project on MS Windows using the lab machines is relatively straightforward after following the instructions provided in Section~\ref{sec:Installing OpenCV} (see Figure~\ref{fig:config proj}). 
	\begin{figure}[tbhp]
	\includegraphics[width=\textwidth]{cmake}
	\caption{\label{fig:config proj}Project configuration using CMake.}
\end{figure}


\newpage
%%%%%%%%%%%%%%%%%%%%%%%%%%%%%%%%%%%%%%%%%%%%%%%%%%%%%%%%%%%%%%%%%%%%%%%%%%%%%%%%
\section{Opening and Displaying an Image}


\subsection{Headers}


\begin{lstlisting}[language=c++,caption=Header files.]
#include <exception> // Header for catching exceptions
#include <iostream>  // Header to display text in the console
#include <opencv2/opencv.hpp> // Main OpenCV header
\end{lstlisting}

OpenCV uses exceptions. To catch them, we need \verb+<exception>+. 
To display text in the console \verb+<iostream>+ is required. 
The main OpenCV header is \verb+<opencv2/opencv.hpp>+.


\subsection{Main structure}

As stated previously, OpenCV uses exceptions. 
We can (or should) catch them to handle errors. 
The structure of the main is shown in Listing~\ref{lst:displayImage}.

\lstinputlisting[language=c++,caption=Initial program to display an image using OpenCV.,label=lst:displayImage,firstline=49,lastline=92,label=lst:displayImage]{../src/displayImage.cxx}


\subsection{Arguments of the Command Line}

The first program only takes one parameter. 
It corresponds to the path of an image file. To make sure the number of arguments is correct, you can use:

 \begin{lstlisting}[language=c++,caption=Checking the number of command line arguments.]
       // No file to display
        if (argc != 2)
        {
            // Create an error message
            std::string error_message;
            error_message  = "usage: ";
            error_message += argv[0];
            error_message += " <input_image>";

            // Throw an error
            throw error_message;
        }
\end{lstlisting}

To get the file name, you can use:
\begin{lstlisting}[language=c++,caption=Getting the file name from the command line arguments.]
std::string input_filename(argv[1]);
\end{lstlisting}

\subsection{Reading the File}

An image is stored in an instance of the class  \verb+Mat+. 
Note that OpenCV's namespace is \verb+cv::+. 
To declare the variable that will hold our image, type:
 \begin{lstlisting}[language=c++]
        // Create an image instance
        cv::Mat image;
\end{lstlisting}

In OpenCV 2 and 3, the image is loaded using:
 \begin{lstlisting}[language=c++,caption=Open an image with OpenCV2 and 3.]
        // Open and read the image
        image = cv::imread( input_filename, CV_LOAD_IMAGE_COLOR );
\end{lstlisting}

In OpenCV 4, the image is loaded using:
 \begin{lstlisting}[language=c++,caption=Open an image with OpenCV4.]
        // Open and read the image
        image = cv::imread( input_filename, cv::IMREAD_COLOR );
\end{lstlisting}

As a consequence, we can use the C pre-porcessor to make sure our code is compatible with either version of OpenCV:
 \begin{lstlisting}[language=c++,caption=Open an image with any version of OpenCV.]
        // Open and read the image
#if CV_MAJOR_VERSION == 2
        image = cv::imread( input_filename, CV_LOAD_IMAGE_COLOR );
#elif CV_MAJOR_VERSION == 3
        image = cv::imread( input_filename, CV_LOAD_IMAGE_COLOR );
#elif CV_MAJOR_VERSION == 4
        image = cv::imread( input_filename, cv::IMREAD_COLOR );
#endif
\end{lstlisting}

It is a good practice to check if any error occurred, e.g.~to avoid unspecified behaviours and crashed. 
If the image is not loaded, its \verb+data+ field is empty. 
If it is the case we can throw an error as follows:
 \begin{lstlisting}[language=c++,caption=Check that the image contains data]
        // The image has not been loaded
        if (!image.data)
        {
            // Create an error message
            std::string error_message;
            error_message  = "Could not open or find the image \"";
            error_message += input_filename;
            error_message += "\".";

            // Throw an error
            throw error_message;
        }
\end{lstlisting}


\subsection{Displaying the Image}

There are four steps to create a window and display and image:
\begin{enumerate}
 \item Create a string to contain the window title (it is used to identify the window);
 \item Create the window;
 \item Show the image in the window;
 \item Wait for a user input to leave the window.
\end{enumerate}

It can be done as follows:
\begin{lstlisting}[language=c++,caption=Create an image.]
        // Create a string to contain the window title
        string window_title;
        window_title  = "Display \"";
        window_title += input_filename;
        window_title += "\"";

        // Create the window
        cv::namedWindow(window_title, cv::WINDOW_AUTOSIZE);

        // Show the image in the window
        cv::imshow(window_title, image);

        // Wait for a user input to leave the window
        cv::waitKey(0);
\end{lstlisting}

The program is now complete. 
You can compile it and run it with different image files to test it.
Figure~\ref{fig:displayImage} shows a screenshot of the program. 

    \begin {figure}[htb]
			\centering
			\includegraphics[width=\textwidth]{displayImage}
      \caption{\label{fig:displayImage}Screenshot of \textbf{displayImage}.}
    \end {figure}

%%%%%%%%%%%%%%%%%%%%%%%%%%%%%%%%%%%%%%%%%%%%%%%%%%%%%%%%%%%%%%%%%%%%%%%%%%%%%%%%
\section{Convert a RGB Image in a Greyscale Image}

Copy the main function of \verb+displayImage.cxx+ into \verb+rgb2grey.cxx+.

\subsection{Arguments of the Command Line}

The second program takes two parameter:
\begin{enumerate}
 \item  The path of the input RGB image file, and
 \item  The path of the output greyscale image file.
\end{enumerate}
Modify the code accordingly. 


\subsection{Converting from RGB to Greyscale}

After displaying the RGB image and BEFORE \verb+cv::waitKey(0)+, create a new image called \verb+grey_image+. 
To convert the original image in greyscale, simply type:
\begin{lstlisting}[language=c++,caption=Convert the colour model of the image in OpenCV2 and 3.]
        // If the image is not a greyscale image, then convert it.
        cv::Mat grey_image;
        cv::cvtColor(image, grey_image, CV_RGB2GRAY);
\end{lstlisting}

\begin{lstlisting}[language=c++,caption=Convert the colour model of the image in OpenCV4.]
        // If the image is not a greyscale image, then convert it.
        cv::Mat grey_image;
        cv::cvtColor(image, grey_image, cv::COLOR_RGB2GRAY);
\end{lstlisting}

In OpenCV in general, the first argument is the input image; the second argument is the output image; other arguments are the parameters of the function. 
Now create another window where to display the new image. 

\subsection{Saving an Image into a File}

The function to save an image is \verb+cv::imwrite(file_name, image)+. 
It returns true if the file has been successfully written; false otherwise. 
We can use the return value to handle possible errors:
\begin{lstlisting}[language=c++,caption=Save an image.]
        // Write the image
        if (!cv::imwrite(argv[2], grey_image))
        {
            // The image has not been written

            // Create an error message
            std::string error_message;
            error_message  = "Could not write the image \"";
            error_message += argv[2];
            error_message += "\".";

            // Throw an error
            throw error_message;
        }
\end{lstlisting}
Calling \verb+rgb2grey lena_color_512.tif lena.png+ should produce the output presented in Figure~\ref{fig:grey}.
    \begin {figure}[htb]
	\centering
	\begin{tabular}{cc}
		\includegraphics[width=0.35\textwidth]{lena_color_512}&
		\includegraphics[width=0.35\textwidth]{lena_grey}\\
		(a) Input image. & (b) Output image.
	\end{tabular}
      \caption{\label{fig:grey}Input and output of \textbf{rgb2grey}.}
    \end {figure}

%%%%%%%%%%%%%%%%%%%%%%%%%%%%%%%%%%%%%%%%%%%%%%%%%%%%%%%%%%%%%%%%%%%%%%%%%%%%%%%%
\section{Mean Filter}

Let us consider the mean filter. 
Copy parts of the main function of \verb+rgb2grey.cxx+ into \verb+meanFilter.cxx+.
The new program will take 3 inputs: 
\begin{enumerate}
	\item The input image;
	\item The output image; and
	\item The convolution kernel radius.
\end{enumerate}
To convert a C string into an integer, use the \verb+atoi+ function from the \verb+<cstdlib>+ header. 
It will be needed to get the kernel radius from the command line argument. 
To set the kernel size, you need to use an instance of the \verb+cv::Size+ class. 
You also have to specify its size. 
You can use:
\begin{lstlisting}[language=c++]
// Filter size
cv::Size filter_size(kernel_width, kernel_height);
\end{lstlisting}
or 
\begin{lstlisting}[language=c++]
// Filter size
cv::Size filter_size;
filter_size.width = kernel_width;
filter_size.height = kernel_height;
\end{lstlisting}
Note that 
\begin{itemize}
	\item If the radius is 0, then the kernel size is $1\times1$    
	\item If the radius is 1, then the kernel size is $3\times3$    
	\item If the radius is 2, then the kernel size is $5\times5$    
	\item ...
	\item If the radius is 7, then the kernel size is $15\times15$    
	\item etc.
\end{itemize}
Now you are ready to filter the input image. Use either \verb+cv::blur+ or \verb+cv::boxFilter+. 
They are the same. 
The first argument is the input image; the second is the output image; and the third one is the kernel size. 
Display and save the output image. 
Try different kernel sizes to see the differences (see Figure~\ref{fig:boxFilter}). 

    \begin {figure}[htb]
	\centering
	\begin{tabular}{cc}
		\includegraphics[width=0.35\textwidth]{lena_R1}&
		\includegraphics[width=0.35\textwidth]{lena_R2}\\
		(a) Radius = 1. & (b) Radius = 2.\\[1em]
		\includegraphics[width=0.35\textwidth]{lena_R5}&
		\includegraphics[width=0.35\textwidth]{lena_R50}\\
		(c) Radius = 5. & (d) Radius = 50.
	\end{tabular}
      \caption{\label{fig:boxFilter}Outputs of \textbf{meanFilter}.}
    \end {figure}


%%%%%%%%%%%%%%%%%%%%%%%%%%%%%%%%%%%%%%%%%%%%%%%%%%%%%%%%%%%%%%%%%%%%%%%%%%%%%%%%
%\section{Median Filter}
%
%You can also use a median filter. 
%In this can, no instance of  \verb+cv::Size+ is required. 
%Just use \verb+cv::medianBlur+. 
%The first argument is the input image; the second is the output image; and the third one is an odd number which corresponds to the kernel size. 
%Try different kernel sizes to see the differences. 


%%%%%%%%%%%%%%%%%%%%%%%%%%%%%%%%%%%%%%%%%%%%%%%%%%%%%%%%%%%%%%%%%%%%%%%%%%%%%%%%
%\section{Gaussian Filter}
%
%Finally, let us try the Gaussian filter. 
%You will need 4 parameters from the command line:
%input file, output file, kernel radius and $\sigma$.
%Just like the mean filter, an instance of \verb+cv::Size+ is needed as the Gaussian filter is a convolution. 
%The first three parameters of \verb+cv::GaussianBlur+ are consistent. 
%The fourth and fifth are the horizontal and vertical $\sigma$ values. 
%Try different kernel sizes and different  $\sigma$ values to see the differences. 


%%%%%%%%%%%%%%%%%%%%%%%%%%%%%%%%%%%%%%%%%%%%%%%%%%%%%%%%%%%%%%%%%%%%%%%%%%%%%%%%
\section{Display an Image in the Log Scale}

The main function of \verb+rgb2grey.cxx+ into \verb+logScale.cxx+ as it is important to use a greyscale image in this new program. 
Fig.~\ref{fig:log} shows the shape of the $\log$ function. 
    \begin {figure}[htb]
      \begin{center}
        % GNUPLOT: LaTeX picture
\setlength{\unitlength}{0.240900pt}
\ifx\plotpoint\undefined\newsavebox{\plotpoint}\fi
\sbox{\plotpoint}{\rule[-0.200pt]{0.400pt}{0.400pt}}%
\begin{picture}(1500,900)(0,0)
\sbox{\plotpoint}{\rule[-0.200pt]{0.400pt}{0.400pt}}%
\put(131.0,131.0){\rule[-0.200pt]{4.818pt}{0.400pt}}
\put(111,131){\makebox(0,0)[r]{-3}}
\put(1419.0,131.0){\rule[-0.200pt]{4.818pt}{0.400pt}}
\put(131.0,235.0){\rule[-0.200pt]{4.818pt}{0.400pt}}
\put(111,235){\makebox(0,0)[r]{-2}}
\put(1419.0,235.0){\rule[-0.200pt]{4.818pt}{0.400pt}}
\put(131.0,339.0){\rule[-0.200pt]{4.818pt}{0.400pt}}
\put(111,339){\makebox(0,0)[r]{-1}}
\put(1419.0,339.0){\rule[-0.200pt]{4.818pt}{0.400pt}}
\put(131.0,443.0){\rule[-0.200pt]{4.818pt}{0.400pt}}
\put(111,443){\makebox(0,0)[r]{ 0}}
\put(1419.0,443.0){\rule[-0.200pt]{4.818pt}{0.400pt}}
\put(131.0,547.0){\rule[-0.200pt]{4.818pt}{0.400pt}}
\put(111,547){\makebox(0,0)[r]{ 1}}
\put(1419.0,547.0){\rule[-0.200pt]{4.818pt}{0.400pt}}
\put(131.0,651.0){\rule[-0.200pt]{4.818pt}{0.400pt}}
\put(111,651){\makebox(0,0)[r]{ 2}}
\put(1419.0,651.0){\rule[-0.200pt]{4.818pt}{0.400pt}}
\put(131.0,755.0){\rule[-0.200pt]{4.818pt}{0.400pt}}
\put(111,755){\makebox(0,0)[r]{ 3}}
\put(1419.0,755.0){\rule[-0.200pt]{4.818pt}{0.400pt}}
\put(131.0,859.0){\rule[-0.200pt]{4.818pt}{0.400pt}}
\put(111,859){\makebox(0,0)[r]{ 4}}
\put(1419.0,859.0){\rule[-0.200pt]{4.818pt}{0.400pt}}
\put(262.0,131.0){\rule[-0.200pt]{0.400pt}{4.818pt}}
\put(262,90){\makebox(0,0){-40}}
\put(262.0,839.0){\rule[-0.200pt]{0.400pt}{4.818pt}}
\put(523.0,131.0){\rule[-0.200pt]{0.400pt}{4.818pt}}
\put(523,90){\makebox(0,0){-20}}
\put(523.0,839.0){\rule[-0.200pt]{0.400pt}{4.818pt}}
\put(785.0,131.0){\rule[-0.200pt]{0.400pt}{4.818pt}}
\put(785,90){\makebox(0,0){ 0}}
\put(785.0,839.0){\rule[-0.200pt]{0.400pt}{4.818pt}}
\put(1047.0,131.0){\rule[-0.200pt]{0.400pt}{4.818pt}}
\put(1047,90){\makebox(0,0){ 20}}
\put(1047.0,839.0){\rule[-0.200pt]{0.400pt}{4.818pt}}
\put(1308.0,131.0){\rule[-0.200pt]{0.400pt}{4.818pt}}
\put(1308,90){\makebox(0,0){ 40}}
\put(1308.0,839.0){\rule[-0.200pt]{0.400pt}{4.818pt}}
\put(131.0,131.0){\rule[-0.200pt]{0.400pt}{175.375pt}}
\put(131.0,131.0){\rule[-0.200pt]{315.097pt}{0.400pt}}
\put(1439.0,131.0){\rule[-0.200pt]{0.400pt}{175.375pt}}
\put(131.0,859.0){\rule[-0.200pt]{315.097pt}{0.400pt}}
\put(30,495){\makebox(0,0){$y = \log{x}$}}
\put(785,29){\makebox(0,0){$x$ axis}}
\put(786,204){\usebox{\plotpoint}}
\multiput(786.61,204.00)(0.447,25.244){3}{\rule{0.108pt}{15.300pt}}
\multiput(785.17,204.00)(3.000,82.244){2}{\rule{0.400pt}{7.650pt}}
\multiput(789.61,318.00)(0.447,11.625){3}{\rule{0.108pt}{7.167pt}}
\multiput(788.17,318.00)(3.000,38.125){2}{\rule{0.400pt}{3.583pt}}
\put(792.17,371){\rule{0.400pt}{7.100pt}}
\multiput(791.17,371.00)(2.000,20.264){2}{\rule{0.400pt}{3.550pt}}
\multiput(794.61,406.00)(0.447,5.597){3}{\rule{0.108pt}{3.567pt}}
\multiput(793.17,406.00)(3.000,18.597){2}{\rule{0.400pt}{1.783pt}}
\put(797.17,432){\rule{0.400pt}{4.300pt}}
\multiput(796.17,432.00)(2.000,12.075){2}{\rule{0.400pt}{2.150pt}}
\multiput(799.61,453.00)(0.447,3.588){3}{\rule{0.108pt}{2.367pt}}
\multiput(798.17,453.00)(3.000,12.088){2}{\rule{0.400pt}{1.183pt}}
\multiput(802.61,470.00)(0.447,3.141){3}{\rule{0.108pt}{2.100pt}}
\multiput(801.17,470.00)(3.000,10.641){2}{\rule{0.400pt}{1.050pt}}
\put(805.17,485){\rule{0.400pt}{2.700pt}}
\multiput(804.17,485.00)(2.000,7.396){2}{\rule{0.400pt}{1.350pt}}
\multiput(807.61,498.00)(0.447,2.472){3}{\rule{0.108pt}{1.700pt}}
\multiput(806.17,498.00)(3.000,8.472){2}{\rule{0.400pt}{0.850pt}}
\multiput(810.61,510.00)(0.447,2.025){3}{\rule{0.108pt}{1.433pt}}
\multiput(809.17,510.00)(3.000,7.025){2}{\rule{0.400pt}{0.717pt}}
\put(813.17,520){\rule{0.400pt}{2.100pt}}
\multiput(812.17,520.00)(2.000,5.641){2}{\rule{0.400pt}{1.050pt}}
\multiput(815.61,530.00)(0.447,1.802){3}{\rule{0.108pt}{1.300pt}}
\multiput(814.17,530.00)(3.000,6.302){2}{\rule{0.400pt}{0.650pt}}
\put(818.17,539){\rule{0.400pt}{1.700pt}}
\multiput(817.17,539.00)(2.000,4.472){2}{\rule{0.400pt}{0.850pt}}
\multiput(820.61,547.00)(0.447,1.355){3}{\rule{0.108pt}{1.033pt}}
\multiput(819.17,547.00)(3.000,4.855){2}{\rule{0.400pt}{0.517pt}}
\multiput(823.61,554.00)(0.447,1.355){3}{\rule{0.108pt}{1.033pt}}
\multiput(822.17,554.00)(3.000,4.855){2}{\rule{0.400pt}{0.517pt}}
\put(826.17,561){\rule{0.400pt}{1.300pt}}
\multiput(825.17,561.00)(2.000,3.302){2}{\rule{0.400pt}{0.650pt}}
\multiput(828.61,567.00)(0.447,1.132){3}{\rule{0.108pt}{0.900pt}}
\multiput(827.17,567.00)(3.000,4.132){2}{\rule{0.400pt}{0.450pt}}
\put(831.17,573){\rule{0.400pt}{1.300pt}}
\multiput(830.17,573.00)(2.000,3.302){2}{\rule{0.400pt}{0.650pt}}
\multiput(833.61,579.00)(0.447,1.132){3}{\rule{0.108pt}{0.900pt}}
\multiput(832.17,579.00)(3.000,4.132){2}{\rule{0.400pt}{0.450pt}}
\multiput(836.61,585.00)(0.447,0.909){3}{\rule{0.108pt}{0.767pt}}
\multiput(835.17,585.00)(3.000,3.409){2}{\rule{0.400pt}{0.383pt}}
\put(839.17,590){\rule{0.400pt}{1.100pt}}
\multiput(838.17,590.00)(2.000,2.717){2}{\rule{0.400pt}{0.550pt}}
\multiput(841.61,595.00)(0.447,0.909){3}{\rule{0.108pt}{0.767pt}}
\multiput(840.17,595.00)(3.000,3.409){2}{\rule{0.400pt}{0.383pt}}
\multiput(844.61,600.00)(0.447,0.685){3}{\rule{0.108pt}{0.633pt}}
\multiput(843.17,600.00)(3.000,2.685){2}{\rule{0.400pt}{0.317pt}}
\put(847.17,604){\rule{0.400pt}{0.900pt}}
\multiput(846.17,604.00)(2.000,2.132){2}{\rule{0.400pt}{0.450pt}}
\multiput(849.61,608.00)(0.447,0.909){3}{\rule{0.108pt}{0.767pt}}
\multiput(848.17,608.00)(3.000,3.409){2}{\rule{0.400pt}{0.383pt}}
\put(852.17,613){\rule{0.400pt}{0.900pt}}
\multiput(851.17,613.00)(2.000,2.132){2}{\rule{0.400pt}{0.450pt}}
\multiput(854.61,617.00)(0.447,0.685){3}{\rule{0.108pt}{0.633pt}}
\multiput(853.17,617.00)(3.000,2.685){2}{\rule{0.400pt}{0.317pt}}
\multiput(857.00,621.61)(0.462,0.447){3}{\rule{0.500pt}{0.108pt}}
\multiput(857.00,620.17)(1.962,3.000){2}{\rule{0.250pt}{0.400pt}}
\put(860.17,624){\rule{0.400pt}{0.900pt}}
\multiput(859.17,624.00)(2.000,2.132){2}{\rule{0.400pt}{0.450pt}}
\multiput(862.00,628.61)(0.462,0.447){3}{\rule{0.500pt}{0.108pt}}
\multiput(862.00,627.17)(1.962,3.000){2}{\rule{0.250pt}{0.400pt}}
\multiput(865.61,631.00)(0.447,0.685){3}{\rule{0.108pt}{0.633pt}}
\multiput(864.17,631.00)(3.000,2.685){2}{\rule{0.400pt}{0.317pt}}
\put(868.17,635){\rule{0.400pt}{0.700pt}}
\multiput(867.17,635.00)(2.000,1.547){2}{\rule{0.400pt}{0.350pt}}
\multiput(870.00,638.61)(0.462,0.447){3}{\rule{0.500pt}{0.108pt}}
\multiput(870.00,637.17)(1.962,3.000){2}{\rule{0.250pt}{0.400pt}}
\put(873.17,641){\rule{0.400pt}{0.700pt}}
\multiput(872.17,641.00)(2.000,1.547){2}{\rule{0.400pt}{0.350pt}}
\multiput(875.00,644.61)(0.462,0.447){3}{\rule{0.500pt}{0.108pt}}
\multiput(875.00,643.17)(1.962,3.000){2}{\rule{0.250pt}{0.400pt}}
\multiput(878.00,647.61)(0.462,0.447){3}{\rule{0.500pt}{0.108pt}}
\multiput(878.00,646.17)(1.962,3.000){2}{\rule{0.250pt}{0.400pt}}
\put(881.17,650){\rule{0.400pt}{0.700pt}}
\multiput(880.17,650.00)(2.000,1.547){2}{\rule{0.400pt}{0.350pt}}
\put(883,653.17){\rule{0.700pt}{0.400pt}}
\multiput(883.00,652.17)(1.547,2.000){2}{\rule{0.350pt}{0.400pt}}
\multiput(886.00,655.61)(0.462,0.447){3}{\rule{0.500pt}{0.108pt}}
\multiput(886.00,654.17)(1.962,3.000){2}{\rule{0.250pt}{0.400pt}}
\put(889.17,658){\rule{0.400pt}{0.700pt}}
\multiput(888.17,658.00)(2.000,1.547){2}{\rule{0.400pt}{0.350pt}}
\put(891,661.17){\rule{0.700pt}{0.400pt}}
\multiput(891.00,660.17)(1.547,2.000){2}{\rule{0.350pt}{0.400pt}}
\put(894.17,663){\rule{0.400pt}{0.700pt}}
\multiput(893.17,663.00)(2.000,1.547){2}{\rule{0.400pt}{0.350pt}}
\put(896,666.17){\rule{0.700pt}{0.400pt}}
\multiput(896.00,665.17)(1.547,2.000){2}{\rule{0.350pt}{0.400pt}}
\multiput(899.00,668.61)(0.462,0.447){3}{\rule{0.500pt}{0.108pt}}
\multiput(899.00,667.17)(1.962,3.000){2}{\rule{0.250pt}{0.400pt}}
\put(902,671.17){\rule{0.482pt}{0.400pt}}
\multiput(902.00,670.17)(1.000,2.000){2}{\rule{0.241pt}{0.400pt}}
\put(904,673.17){\rule{0.700pt}{0.400pt}}
\multiput(904.00,672.17)(1.547,2.000){2}{\rule{0.350pt}{0.400pt}}
\put(907,675.17){\rule{0.700pt}{0.400pt}}
\multiput(907.00,674.17)(1.547,2.000){2}{\rule{0.350pt}{0.400pt}}
\put(910.17,677){\rule{0.400pt}{0.700pt}}
\multiput(909.17,677.00)(2.000,1.547){2}{\rule{0.400pt}{0.350pt}}
\put(912,680.17){\rule{0.700pt}{0.400pt}}
\multiput(912.00,679.17)(1.547,2.000){2}{\rule{0.350pt}{0.400pt}}
\put(915,682.17){\rule{0.482pt}{0.400pt}}
\multiput(915.00,681.17)(1.000,2.000){2}{\rule{0.241pt}{0.400pt}}
\put(917,684.17){\rule{0.700pt}{0.400pt}}
\multiput(917.00,683.17)(1.547,2.000){2}{\rule{0.350pt}{0.400pt}}
\put(920,686.17){\rule{0.700pt}{0.400pt}}
\multiput(920.00,685.17)(1.547,2.000){2}{\rule{0.350pt}{0.400pt}}
\put(923,688.17){\rule{0.482pt}{0.400pt}}
\multiput(923.00,687.17)(1.000,2.000){2}{\rule{0.241pt}{0.400pt}}
\put(925,690.17){\rule{0.700pt}{0.400pt}}
\multiput(925.00,689.17)(1.547,2.000){2}{\rule{0.350pt}{0.400pt}}
\put(928,692.17){\rule{0.482pt}{0.400pt}}
\multiput(928.00,691.17)(1.000,2.000){2}{\rule{0.241pt}{0.400pt}}
\put(930,693.67){\rule{0.723pt}{0.400pt}}
\multiput(930.00,693.17)(1.500,1.000){2}{\rule{0.361pt}{0.400pt}}
\put(933,695.17){\rule{0.700pt}{0.400pt}}
\multiput(933.00,694.17)(1.547,2.000){2}{\rule{0.350pt}{0.400pt}}
\put(936,697.17){\rule{0.482pt}{0.400pt}}
\multiput(936.00,696.17)(1.000,2.000){2}{\rule{0.241pt}{0.400pt}}
\put(938,699.17){\rule{0.700pt}{0.400pt}}
\multiput(938.00,698.17)(1.547,2.000){2}{\rule{0.350pt}{0.400pt}}
\put(941,701.17){\rule{0.700pt}{0.400pt}}
\multiput(941.00,700.17)(1.547,2.000){2}{\rule{0.350pt}{0.400pt}}
\put(944,702.67){\rule{0.482pt}{0.400pt}}
\multiput(944.00,702.17)(1.000,1.000){2}{\rule{0.241pt}{0.400pt}}
\put(946,704.17){\rule{0.700pt}{0.400pt}}
\multiput(946.00,703.17)(1.547,2.000){2}{\rule{0.350pt}{0.400pt}}
\put(949,706.17){\rule{0.482pt}{0.400pt}}
\multiput(949.00,705.17)(1.000,2.000){2}{\rule{0.241pt}{0.400pt}}
\put(951,707.67){\rule{0.723pt}{0.400pt}}
\multiput(951.00,707.17)(1.500,1.000){2}{\rule{0.361pt}{0.400pt}}
\put(954,709.17){\rule{0.700pt}{0.400pt}}
\multiput(954.00,708.17)(1.547,2.000){2}{\rule{0.350pt}{0.400pt}}
\put(957,710.67){\rule{0.482pt}{0.400pt}}
\multiput(957.00,710.17)(1.000,1.000){2}{\rule{0.241pt}{0.400pt}}
\put(959,712.17){\rule{0.700pt}{0.400pt}}
\multiput(959.00,711.17)(1.547,2.000){2}{\rule{0.350pt}{0.400pt}}
\put(962,713.67){\rule{0.723pt}{0.400pt}}
\multiput(962.00,713.17)(1.500,1.000){2}{\rule{0.361pt}{0.400pt}}
\put(965,715.17){\rule{0.482pt}{0.400pt}}
\multiput(965.00,714.17)(1.000,2.000){2}{\rule{0.241pt}{0.400pt}}
\put(967,716.67){\rule{0.723pt}{0.400pt}}
\multiput(967.00,716.17)(1.500,1.000){2}{\rule{0.361pt}{0.400pt}}
\put(970,718.17){\rule{0.482pt}{0.400pt}}
\multiput(970.00,717.17)(1.000,2.000){2}{\rule{0.241pt}{0.400pt}}
\put(972,719.67){\rule{0.723pt}{0.400pt}}
\multiput(972.00,719.17)(1.500,1.000){2}{\rule{0.361pt}{0.400pt}}
\put(975,721.17){\rule{0.700pt}{0.400pt}}
\multiput(975.00,720.17)(1.547,2.000){2}{\rule{0.350pt}{0.400pt}}
\put(978,722.67){\rule{0.482pt}{0.400pt}}
\multiput(978.00,722.17)(1.000,1.000){2}{\rule{0.241pt}{0.400pt}}
\put(980,724.17){\rule{0.700pt}{0.400pt}}
\multiput(980.00,723.17)(1.547,2.000){2}{\rule{0.350pt}{0.400pt}}
\put(983,725.67){\rule{0.723pt}{0.400pt}}
\multiput(983.00,725.17)(1.500,1.000){2}{\rule{0.361pt}{0.400pt}}
\put(986,726.67){\rule{0.482pt}{0.400pt}}
\multiput(986.00,726.17)(1.000,1.000){2}{\rule{0.241pt}{0.400pt}}
\put(988,728.17){\rule{0.700pt}{0.400pt}}
\multiput(988.00,727.17)(1.547,2.000){2}{\rule{0.350pt}{0.400pt}}
\put(991,729.67){\rule{0.482pt}{0.400pt}}
\multiput(991.00,729.17)(1.000,1.000){2}{\rule{0.241pt}{0.400pt}}
\put(993,730.67){\rule{0.723pt}{0.400pt}}
\multiput(993.00,730.17)(1.500,1.000){2}{\rule{0.361pt}{0.400pt}}
\put(996,731.67){\rule{0.723pt}{0.400pt}}
\multiput(996.00,731.17)(1.500,1.000){2}{\rule{0.361pt}{0.400pt}}
\put(999,733.17){\rule{0.482pt}{0.400pt}}
\multiput(999.00,732.17)(1.000,2.000){2}{\rule{0.241pt}{0.400pt}}
\put(1001,734.67){\rule{0.723pt}{0.400pt}}
\multiput(1001.00,734.17)(1.500,1.000){2}{\rule{0.361pt}{0.400pt}}
\put(1004,735.67){\rule{0.482pt}{0.400pt}}
\multiput(1004.00,735.17)(1.000,1.000){2}{\rule{0.241pt}{0.400pt}}
\put(1006,736.67){\rule{0.723pt}{0.400pt}}
\multiput(1006.00,736.17)(1.500,1.000){2}{\rule{0.361pt}{0.400pt}}
\put(1009,738.17){\rule{0.700pt}{0.400pt}}
\multiput(1009.00,737.17)(1.547,2.000){2}{\rule{0.350pt}{0.400pt}}
\put(1012,739.67){\rule{0.482pt}{0.400pt}}
\multiput(1012.00,739.17)(1.000,1.000){2}{\rule{0.241pt}{0.400pt}}
\put(1014,740.67){\rule{0.723pt}{0.400pt}}
\multiput(1014.00,740.17)(1.500,1.000){2}{\rule{0.361pt}{0.400pt}}
\put(1017,741.67){\rule{0.723pt}{0.400pt}}
\multiput(1017.00,741.17)(1.500,1.000){2}{\rule{0.361pt}{0.400pt}}
\put(1020,742.67){\rule{0.482pt}{0.400pt}}
\multiput(1020.00,742.17)(1.000,1.000){2}{\rule{0.241pt}{0.400pt}}
\put(1022,744.17){\rule{0.700pt}{0.400pt}}
\multiput(1022.00,743.17)(1.547,2.000){2}{\rule{0.350pt}{0.400pt}}
\put(1025,745.67){\rule{0.482pt}{0.400pt}}
\multiput(1025.00,745.17)(1.000,1.000){2}{\rule{0.241pt}{0.400pt}}
\put(1027,746.67){\rule{0.723pt}{0.400pt}}
\multiput(1027.00,746.17)(1.500,1.000){2}{\rule{0.361pt}{0.400pt}}
\put(1030,747.67){\rule{0.723pt}{0.400pt}}
\multiput(1030.00,747.17)(1.500,1.000){2}{\rule{0.361pt}{0.400pt}}
\put(1033,748.67){\rule{0.482pt}{0.400pt}}
\multiput(1033.00,748.17)(1.000,1.000){2}{\rule{0.241pt}{0.400pt}}
\put(1035,749.67){\rule{0.723pt}{0.400pt}}
\multiput(1035.00,749.17)(1.500,1.000){2}{\rule{0.361pt}{0.400pt}}
\put(1038,750.67){\rule{0.723pt}{0.400pt}}
\multiput(1038.00,750.17)(1.500,1.000){2}{\rule{0.361pt}{0.400pt}}
\put(1041,751.67){\rule{0.482pt}{0.400pt}}
\multiput(1041.00,751.17)(1.000,1.000){2}{\rule{0.241pt}{0.400pt}}
\put(1043,752.67){\rule{0.723pt}{0.400pt}}
\multiput(1043.00,752.17)(1.500,1.000){2}{\rule{0.361pt}{0.400pt}}
\put(1046,753.67){\rule{0.482pt}{0.400pt}}
\multiput(1046.00,753.17)(1.000,1.000){2}{\rule{0.241pt}{0.400pt}}
\put(1048,754.67){\rule{0.723pt}{0.400pt}}
\multiput(1048.00,754.17)(1.500,1.000){2}{\rule{0.361pt}{0.400pt}}
\put(1051,755.67){\rule{0.723pt}{0.400pt}}
\multiput(1051.00,755.17)(1.500,1.000){2}{\rule{0.361pt}{0.400pt}}
\put(1054,756.67){\rule{0.482pt}{0.400pt}}
\multiput(1054.00,756.17)(1.000,1.000){2}{\rule{0.241pt}{0.400pt}}
\put(1056,757.67){\rule{0.723pt}{0.400pt}}
\multiput(1056.00,757.17)(1.500,1.000){2}{\rule{0.361pt}{0.400pt}}
\put(1059,758.67){\rule{0.723pt}{0.400pt}}
\multiput(1059.00,758.17)(1.500,1.000){2}{\rule{0.361pt}{0.400pt}}
\put(1062,759.67){\rule{0.482pt}{0.400pt}}
\multiput(1062.00,759.17)(1.000,1.000){2}{\rule{0.241pt}{0.400pt}}
\put(1064,760.67){\rule{0.723pt}{0.400pt}}
\multiput(1064.00,760.17)(1.500,1.000){2}{\rule{0.361pt}{0.400pt}}
\put(1067,761.67){\rule{0.482pt}{0.400pt}}
\multiput(1067.00,761.17)(1.000,1.000){2}{\rule{0.241pt}{0.400pt}}
\put(1069,762.67){\rule{0.723pt}{0.400pt}}
\multiput(1069.00,762.17)(1.500,1.000){2}{\rule{0.361pt}{0.400pt}}
\put(1072,763.67){\rule{0.723pt}{0.400pt}}
\multiput(1072.00,763.17)(1.500,1.000){2}{\rule{0.361pt}{0.400pt}}
\put(1075,764.67){\rule{0.482pt}{0.400pt}}
\multiput(1075.00,764.17)(1.000,1.000){2}{\rule{0.241pt}{0.400pt}}
\put(1077,765.67){\rule{0.723pt}{0.400pt}}
\multiput(1077.00,765.17)(1.500,1.000){2}{\rule{0.361pt}{0.400pt}}
\put(1080,766.67){\rule{0.723pt}{0.400pt}}
\multiput(1080.00,766.17)(1.500,1.000){2}{\rule{0.361pt}{0.400pt}}
\put(1083,767.67){\rule{0.482pt}{0.400pt}}
\multiput(1083.00,767.17)(1.000,1.000){2}{\rule{0.241pt}{0.400pt}}
\put(1085,768.67){\rule{0.723pt}{0.400pt}}
\multiput(1085.00,768.17)(1.500,1.000){2}{\rule{0.361pt}{0.400pt}}
\put(1088,769.67){\rule{0.482pt}{0.400pt}}
\multiput(1088.00,769.17)(1.000,1.000){2}{\rule{0.241pt}{0.400pt}}
\put(1090,770.67){\rule{0.723pt}{0.400pt}}
\multiput(1090.00,770.17)(1.500,1.000){2}{\rule{0.361pt}{0.400pt}}
\put(1096,771.67){\rule{0.482pt}{0.400pt}}
\multiput(1096.00,771.17)(1.000,1.000){2}{\rule{0.241pt}{0.400pt}}
\put(1098,772.67){\rule{0.723pt}{0.400pt}}
\multiput(1098.00,772.17)(1.500,1.000){2}{\rule{0.361pt}{0.400pt}}
\put(1101,773.67){\rule{0.482pt}{0.400pt}}
\multiput(1101.00,773.17)(1.000,1.000){2}{\rule{0.241pt}{0.400pt}}
\put(1103,774.67){\rule{0.723pt}{0.400pt}}
\multiput(1103.00,774.17)(1.500,1.000){2}{\rule{0.361pt}{0.400pt}}
\put(1106,775.67){\rule{0.723pt}{0.400pt}}
\multiput(1106.00,775.17)(1.500,1.000){2}{\rule{0.361pt}{0.400pt}}
\put(1109,776.67){\rule{0.482pt}{0.400pt}}
\multiput(1109.00,776.17)(1.000,1.000){2}{\rule{0.241pt}{0.400pt}}
\put(1093.0,772.0){\rule[-0.200pt]{0.723pt}{0.400pt}}
\put(1114,777.67){\rule{0.723pt}{0.400pt}}
\multiput(1114.00,777.17)(1.500,1.000){2}{\rule{0.361pt}{0.400pt}}
\put(1117,778.67){\rule{0.482pt}{0.400pt}}
\multiput(1117.00,778.17)(1.000,1.000){2}{\rule{0.241pt}{0.400pt}}
\put(1119,779.67){\rule{0.723pt}{0.400pt}}
\multiput(1119.00,779.17)(1.500,1.000){2}{\rule{0.361pt}{0.400pt}}
\put(1122,780.67){\rule{0.482pt}{0.400pt}}
\multiput(1122.00,780.17)(1.000,1.000){2}{\rule{0.241pt}{0.400pt}}
\put(1111.0,778.0){\rule[-0.200pt]{0.723pt}{0.400pt}}
\put(1127,781.67){\rule{0.723pt}{0.400pt}}
\multiput(1127.00,781.17)(1.500,1.000){2}{\rule{0.361pt}{0.400pt}}
\put(1130,782.67){\rule{0.482pt}{0.400pt}}
\multiput(1130.00,782.17)(1.000,1.000){2}{\rule{0.241pt}{0.400pt}}
\put(1132,783.67){\rule{0.723pt}{0.400pt}}
\multiput(1132.00,783.17)(1.500,1.000){2}{\rule{0.361pt}{0.400pt}}
\put(1135,784.67){\rule{0.723pt}{0.400pt}}
\multiput(1135.00,784.17)(1.500,1.000){2}{\rule{0.361pt}{0.400pt}}
\put(1124.0,782.0){\rule[-0.200pt]{0.723pt}{0.400pt}}
\put(1140,785.67){\rule{0.723pt}{0.400pt}}
\multiput(1140.00,785.17)(1.500,1.000){2}{\rule{0.361pt}{0.400pt}}
\put(1143,786.67){\rule{0.482pt}{0.400pt}}
\multiput(1143.00,786.17)(1.000,1.000){2}{\rule{0.241pt}{0.400pt}}
\put(1145,787.67){\rule{0.723pt}{0.400pt}}
\multiput(1145.00,787.17)(1.500,1.000){2}{\rule{0.361pt}{0.400pt}}
\put(1138.0,786.0){\rule[-0.200pt]{0.482pt}{0.400pt}}
\put(1151,788.67){\rule{0.482pt}{0.400pt}}
\multiput(1151.00,788.17)(1.000,1.000){2}{\rule{0.241pt}{0.400pt}}
\put(1153,789.67){\rule{0.723pt}{0.400pt}}
\multiput(1153.00,789.17)(1.500,1.000){2}{\rule{0.361pt}{0.400pt}}
\put(1156,790.67){\rule{0.723pt}{0.400pt}}
\multiput(1156.00,790.17)(1.500,1.000){2}{\rule{0.361pt}{0.400pt}}
\put(1148.0,789.0){\rule[-0.200pt]{0.723pt}{0.400pt}}
\put(1161,791.67){\rule{0.723pt}{0.400pt}}
\multiput(1161.00,791.17)(1.500,1.000){2}{\rule{0.361pt}{0.400pt}}
\put(1164,792.67){\rule{0.482pt}{0.400pt}}
\multiput(1164.00,792.17)(1.000,1.000){2}{\rule{0.241pt}{0.400pt}}
\put(1159.0,792.0){\rule[-0.200pt]{0.482pt}{0.400pt}}
\put(1169,793.67){\rule{0.723pt}{0.400pt}}
\multiput(1169.00,793.17)(1.500,1.000){2}{\rule{0.361pt}{0.400pt}}
\put(1172,794.67){\rule{0.482pt}{0.400pt}}
\multiput(1172.00,794.17)(1.000,1.000){2}{\rule{0.241pt}{0.400pt}}
\put(1174,795.67){\rule{0.723pt}{0.400pt}}
\multiput(1174.00,795.17)(1.500,1.000){2}{\rule{0.361pt}{0.400pt}}
\put(1166.0,794.0){\rule[-0.200pt]{0.723pt}{0.400pt}}
\put(1179,796.67){\rule{0.723pt}{0.400pt}}
\multiput(1179.00,796.17)(1.500,1.000){2}{\rule{0.361pt}{0.400pt}}
\put(1182,797.67){\rule{0.723pt}{0.400pt}}
\multiput(1182.00,797.17)(1.500,1.000){2}{\rule{0.361pt}{0.400pt}}
\put(1177.0,797.0){\rule[-0.200pt]{0.482pt}{0.400pt}}
\put(1187,798.67){\rule{0.723pt}{0.400pt}}
\multiput(1187.00,798.17)(1.500,1.000){2}{\rule{0.361pt}{0.400pt}}
\put(1190,799.67){\rule{0.723pt}{0.400pt}}
\multiput(1190.00,799.17)(1.500,1.000){2}{\rule{0.361pt}{0.400pt}}
\put(1185.0,799.0){\rule[-0.200pt]{0.482pt}{0.400pt}}
\put(1195,800.67){\rule{0.723pt}{0.400pt}}
\multiput(1195.00,800.17)(1.500,1.000){2}{\rule{0.361pt}{0.400pt}}
\put(1198,801.67){\rule{0.482pt}{0.400pt}}
\multiput(1198.00,801.17)(1.000,1.000){2}{\rule{0.241pt}{0.400pt}}
\put(1193.0,801.0){\rule[-0.200pt]{0.482pt}{0.400pt}}
\put(1203,802.67){\rule{0.723pt}{0.400pt}}
\multiput(1203.00,802.17)(1.500,1.000){2}{\rule{0.361pt}{0.400pt}}
\put(1206,803.67){\rule{0.482pt}{0.400pt}}
\multiput(1206.00,803.17)(1.000,1.000){2}{\rule{0.241pt}{0.400pt}}
\put(1200.0,803.0){\rule[-0.200pt]{0.723pt}{0.400pt}}
\put(1211,804.67){\rule{0.723pt}{0.400pt}}
\multiput(1211.00,804.17)(1.500,1.000){2}{\rule{0.361pt}{0.400pt}}
\put(1214,805.67){\rule{0.482pt}{0.400pt}}
\multiput(1214.00,805.17)(1.000,1.000){2}{\rule{0.241pt}{0.400pt}}
\put(1208.0,805.0){\rule[-0.200pt]{0.723pt}{0.400pt}}
\put(1219,806.67){\rule{0.482pt}{0.400pt}}
\multiput(1219.00,806.17)(1.000,1.000){2}{\rule{0.241pt}{0.400pt}}
\put(1216.0,807.0){\rule[-0.200pt]{0.723pt}{0.400pt}}
\put(1224,807.67){\rule{0.723pt}{0.400pt}}
\multiput(1224.00,807.17)(1.500,1.000){2}{\rule{0.361pt}{0.400pt}}
\put(1227,808.67){\rule{0.482pt}{0.400pt}}
\multiput(1227.00,808.17)(1.000,1.000){2}{\rule{0.241pt}{0.400pt}}
\put(1221.0,808.0){\rule[-0.200pt]{0.723pt}{0.400pt}}
\put(1232,809.67){\rule{0.723pt}{0.400pt}}
\multiput(1232.00,809.17)(1.500,1.000){2}{\rule{0.361pt}{0.400pt}}
\put(1229.0,810.0){\rule[-0.200pt]{0.723pt}{0.400pt}}
\put(1237,810.67){\rule{0.723pt}{0.400pt}}
\multiput(1237.00,810.17)(1.500,1.000){2}{\rule{0.361pt}{0.400pt}}
\put(1240,811.67){\rule{0.482pt}{0.400pt}}
\multiput(1240.00,811.17)(1.000,1.000){2}{\rule{0.241pt}{0.400pt}}
\put(1235.0,811.0){\rule[-0.200pt]{0.482pt}{0.400pt}}
\put(1245,812.67){\rule{0.723pt}{0.400pt}}
\multiput(1245.00,812.17)(1.500,1.000){2}{\rule{0.361pt}{0.400pt}}
\put(1242.0,813.0){\rule[-0.200pt]{0.723pt}{0.400pt}}
\put(1250,813.67){\rule{0.723pt}{0.400pt}}
\multiput(1250.00,813.17)(1.500,1.000){2}{\rule{0.361pt}{0.400pt}}
\put(1253,814.67){\rule{0.723pt}{0.400pt}}
\multiput(1253.00,814.17)(1.500,1.000){2}{\rule{0.361pt}{0.400pt}}
\put(1248.0,814.0){\rule[-0.200pt]{0.482pt}{0.400pt}}
\put(1258,815.67){\rule{0.723pt}{0.400pt}}
\multiput(1258.00,815.17)(1.500,1.000){2}{\rule{0.361pt}{0.400pt}}
\put(1256.0,816.0){\rule[-0.200pt]{0.482pt}{0.400pt}}
\put(1263,816.67){\rule{0.723pt}{0.400pt}}
\multiput(1263.00,816.17)(1.500,1.000){2}{\rule{0.361pt}{0.400pt}}
\put(1261.0,817.0){\rule[-0.200pt]{0.482pt}{0.400pt}}
\put(1269,817.67){\rule{0.482pt}{0.400pt}}
\multiput(1269.00,817.17)(1.000,1.000){2}{\rule{0.241pt}{0.400pt}}
\put(1271,818.67){\rule{0.723pt}{0.400pt}}
\multiput(1271.00,818.17)(1.500,1.000){2}{\rule{0.361pt}{0.400pt}}
\put(1266.0,818.0){\rule[-0.200pt]{0.723pt}{0.400pt}}
\put(1276,819.67){\rule{0.723pt}{0.400pt}}
\multiput(1276.00,819.17)(1.500,1.000){2}{\rule{0.361pt}{0.400pt}}
\put(1274.0,820.0){\rule[-0.200pt]{0.482pt}{0.400pt}}
\put(1282,820.67){\rule{0.482pt}{0.400pt}}
\multiput(1282.00,820.17)(1.000,1.000){2}{\rule{0.241pt}{0.400pt}}
\put(1279.0,821.0){\rule[-0.200pt]{0.723pt}{0.400pt}}
\put(1287,821.67){\rule{0.723pt}{0.400pt}}
\multiput(1287.00,821.17)(1.500,1.000){2}{\rule{0.361pt}{0.400pt}}
\put(1284.0,822.0){\rule[-0.200pt]{0.723pt}{0.400pt}}
\put(1292,822.67){\rule{0.723pt}{0.400pt}}
\multiput(1292.00,822.17)(1.500,1.000){2}{\rule{0.361pt}{0.400pt}}
\put(1290.0,823.0){\rule[-0.200pt]{0.482pt}{0.400pt}}
\put(1297,823.67){\rule{0.723pt}{0.400pt}}
\multiput(1297.00,823.17)(1.500,1.000){2}{\rule{0.361pt}{0.400pt}}
\put(1300,824.67){\rule{0.723pt}{0.400pt}}
\multiput(1300.00,824.17)(1.500,1.000){2}{\rule{0.361pt}{0.400pt}}
\put(1295.0,824.0){\rule[-0.200pt]{0.482pt}{0.400pt}}
\put(1305,825.67){\rule{0.723pt}{0.400pt}}
\multiput(1305.00,825.17)(1.500,1.000){2}{\rule{0.361pt}{0.400pt}}
\put(1303.0,826.0){\rule[-0.200pt]{0.482pt}{0.400pt}}
\put(1311,826.67){\rule{0.482pt}{0.400pt}}
\multiput(1311.00,826.17)(1.000,1.000){2}{\rule{0.241pt}{0.400pt}}
\put(1308.0,827.0){\rule[-0.200pt]{0.723pt}{0.400pt}}
\put(1316,827.67){\rule{0.482pt}{0.400pt}}
\multiput(1316.00,827.17)(1.000,1.000){2}{\rule{0.241pt}{0.400pt}}
\put(1313.0,828.0){\rule[-0.200pt]{0.723pt}{0.400pt}}
\put(1321,828.67){\rule{0.723pt}{0.400pt}}
\multiput(1321.00,828.17)(1.500,1.000){2}{\rule{0.361pt}{0.400pt}}
\put(1318.0,829.0){\rule[-0.200pt]{0.723pt}{0.400pt}}
\put(1326,829.67){\rule{0.723pt}{0.400pt}}
\multiput(1326.00,829.17)(1.500,1.000){2}{\rule{0.361pt}{0.400pt}}
\put(1324.0,830.0){\rule[-0.200pt]{0.482pt}{0.400pt}}
\put(1332,830.67){\rule{0.482pt}{0.400pt}}
\multiput(1332.00,830.17)(1.000,1.000){2}{\rule{0.241pt}{0.400pt}}
\put(1329.0,831.0){\rule[-0.200pt]{0.723pt}{0.400pt}}
\put(1337,831.67){\rule{0.482pt}{0.400pt}}
\multiput(1337.00,831.17)(1.000,1.000){2}{\rule{0.241pt}{0.400pt}}
\put(1334.0,832.0){\rule[-0.200pt]{0.723pt}{0.400pt}}
\put(1342,832.67){\rule{0.723pt}{0.400pt}}
\multiput(1342.00,832.17)(1.500,1.000){2}{\rule{0.361pt}{0.400pt}}
\put(1339.0,833.0){\rule[-0.200pt]{0.723pt}{0.400pt}}
\put(1347,833.67){\rule{0.723pt}{0.400pt}}
\multiput(1347.00,833.17)(1.500,1.000){2}{\rule{0.361pt}{0.400pt}}
\put(1345.0,834.0){\rule[-0.200pt]{0.482pt}{0.400pt}}
\put(1352,834.67){\rule{0.723pt}{0.400pt}}
\multiput(1352.00,834.17)(1.500,1.000){2}{\rule{0.361pt}{0.400pt}}
\put(1350.0,835.0){\rule[-0.200pt]{0.482pt}{0.400pt}}
\put(1358,835.67){\rule{0.482pt}{0.400pt}}
\multiput(1358.00,835.17)(1.000,1.000){2}{\rule{0.241pt}{0.400pt}}
\put(1355.0,836.0){\rule[-0.200pt]{0.723pt}{0.400pt}}
\put(1366,836.67){\rule{0.482pt}{0.400pt}}
\multiput(1366.00,836.17)(1.000,1.000){2}{\rule{0.241pt}{0.400pt}}
\put(1360.0,837.0){\rule[-0.200pt]{1.445pt}{0.400pt}}
\put(1371,837.67){\rule{0.482pt}{0.400pt}}
\multiput(1371.00,837.17)(1.000,1.000){2}{\rule{0.241pt}{0.400pt}}
\put(1368.0,838.0){\rule[-0.200pt]{0.723pt}{0.400pt}}
\put(1376,838.67){\rule{0.723pt}{0.400pt}}
\multiput(1376.00,838.17)(1.500,1.000){2}{\rule{0.361pt}{0.400pt}}
\put(1373.0,839.0){\rule[-0.200pt]{0.723pt}{0.400pt}}
\put(1381,839.67){\rule{0.723pt}{0.400pt}}
\multiput(1381.00,839.17)(1.500,1.000){2}{\rule{0.361pt}{0.400pt}}
\put(1379.0,840.0){\rule[-0.200pt]{0.482pt}{0.400pt}}
\put(1387,840.67){\rule{0.482pt}{0.400pt}}
\multiput(1387.00,840.17)(1.000,1.000){2}{\rule{0.241pt}{0.400pt}}
\put(1384.0,841.0){\rule[-0.200pt]{0.723pt}{0.400pt}}
\put(1392,841.67){\rule{0.482pt}{0.400pt}}
\multiput(1392.00,841.17)(1.000,1.000){2}{\rule{0.241pt}{0.400pt}}
\put(1389.0,842.0){\rule[-0.200pt]{0.723pt}{0.400pt}}
\put(1400,842.67){\rule{0.482pt}{0.400pt}}
\multiput(1400.00,842.17)(1.000,1.000){2}{\rule{0.241pt}{0.400pt}}
\put(1394.0,843.0){\rule[-0.200pt]{1.445pt}{0.400pt}}
\put(1405,843.67){\rule{0.723pt}{0.400pt}}
\multiput(1405.00,843.17)(1.500,1.000){2}{\rule{0.361pt}{0.400pt}}
\put(1402.0,844.0){\rule[-0.200pt]{0.723pt}{0.400pt}}
\put(1410,844.67){\rule{0.723pt}{0.400pt}}
\multiput(1410.00,844.17)(1.500,1.000){2}{\rule{0.361pt}{0.400pt}}
\put(1408.0,845.0){\rule[-0.200pt]{0.482pt}{0.400pt}}
\put(1418,845.67){\rule{0.723pt}{0.400pt}}
\multiput(1418.00,845.17)(1.500,1.000){2}{\rule{0.361pt}{0.400pt}}
\put(1413.0,846.0){\rule[-0.200pt]{1.204pt}{0.400pt}}
\put(1423,846.67){\rule{0.723pt}{0.400pt}}
\multiput(1423.00,846.17)(1.500,1.000){2}{\rule{0.361pt}{0.400pt}}
\put(1421.0,847.0){\rule[-0.200pt]{0.482pt}{0.400pt}}
\put(1429,847.67){\rule{0.482pt}{0.400pt}}
\multiput(1429.00,847.17)(1.000,1.000){2}{\rule{0.241pt}{0.400pt}}
\put(1426.0,848.0){\rule[-0.200pt]{0.723pt}{0.400pt}}
\put(1436,848.67){\rule{0.723pt}{0.400pt}}
\multiput(1436.00,848.17)(1.500,1.000){2}{\rule{0.361pt}{0.400pt}}
\put(1431.0,849.0){\rule[-0.200pt]{1.204pt}{0.400pt}}
\put(131.0,131.0){\rule[-0.200pt]{0.400pt}{175.375pt}}
\put(131.0,131.0){\rule[-0.200pt]{315.097pt}{0.400pt}}
\put(1439.0,131.0){\rule[-0.200pt]{0.400pt}{175.375pt}}
\put(131.0,859.0){\rule[-0.200pt]{315.097pt}{0.400pt}}
\end{picture}

      \end{center}
      \caption{\label{fig:log}The $\log$ function.}
    \end {figure}
Looking at the $y$ axis, we note that it is important to store the image using floating point numbers. 
If we don't, there will be enormous quantisation problems. 

To convert the greyscale image from \verb+unsigned char+ to \verb+float+, we use:
\begin{lstlisting}[language=c++,caption=Convert an image into floating point numbers.]
        // Convert to float
        cv::Mat float_image;
        grey_image.convertTo(float_image, CV_32FC1);
\end{lstlisting}

It can be seen on the figure that $\log(x) \forall x \in ]-\infty,  0]$ is undefined. 
In other word, if $x$ is equal to zero or $x$ is negative, then there is no $y$ value. 
As the input image was using \verb+unsigned char+, we do not have to worry about negative values. 
However, we have to make sure no $0$ value is present in the image. 
To do so, we apply the following transformation:
\begin{equation}f'(x,y) = \log(f(x,y) + 1)\end{equation}
using
\begin{lstlisting}[language=c++]
        // Log transformation
        cv::Mat log_image;
        cv::log(float_image + 1.0, log_image);
\end{lstlisting}

Looking at the curve, we notice another problem. 
In some case, $\log(x)$ may be negative. 
In this case, it is common to normalise the image so that its values lie in the range $[0, 1]$ using:
\begin{equation}f''(x,y) = \frac{f'(x,y) - \min(f')}{\max(f') - \min(f')}\label{eq:normal}\end{equation}

There are two ways to achieve this in OpenCV.
You can implement Eq.~\ref{eq:normal} using:
\begin{lstlisting}[language=c++]
        double min, max;
        cv::minMaxLoc(log_image, &min, &max);
        cv::Mat normalised_image = 255.0 * (log_image - min) / (max - min);
        normalised_image.convertTo(normalised_image, CV_8UC1);
\end{lstlisting}
or you can use OpenCV's function:
\begin{lstlisting}[language=c++]
        // Normalisation
        cv::Mat normalised_image;
        cv::normalize(log_image, normalised_image, 0, 255, cv::NORM_MINMAX, CV_8UC1);
\end{lstlisting}

Now you can display and save the image (see Figure~\ref{fig:logFilter}.
    
        \begin {figure}[htb]
	\centering
	\begin{tabular}{cc}
		\includegraphics[width=0.35\textwidth]{lake}&
		\includegraphics[width=0.35\textwidth]{lake_log}\\
		(a) Input image. & (b) Output image.
	\end{tabular}
      \caption{\label{fig:logFilter}Input and output of \textbf{logScale}.}
    \end {figure}

    
\section{Additional tasks}

Investigate the use of the median and Gaussian filters. 
The help page for image filtering is availalbe at \url{https://docs.opencv.org/4.0.1/d4/d86/group__imgproc__filter.html}

\begin{itemize}
	\item For \verb|cv::medianBlur|, look at\\ \url{https://docs.opencv.org/4.0.1/d4/d86/group__imgproc__filter.html#ga564869aa33e58769b4469101aac458f9} 
	\item For \verb|cv::GaussianBlur|, look at \url{https://docs.opencv.org/4.0.1/d4/d86/group__imgproc__filter.html#gaabe8c836e97159a9193fb0b11ac52cf1}
\end{itemize}


%%%%%%%%%%%%%%%%%%%%%%%%%%%%%%%%%%%%%%%%%%%%%%%%%%%%%%%%%%%%%%%%%%%%%%%%%%%%%%%%
\end{document}
